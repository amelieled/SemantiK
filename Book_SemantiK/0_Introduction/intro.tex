L'objectif de ce document est de présenter comme représenter des sémantiques dans différents langages existants, à savoir OCaml, Coq, Dedukti et $\mathbb{K}$. \\
	Pour ce faire, nous nous inspirons du cours de sémantique donné à l'ENSIIE, disponible dans le dossier PROG1\_lesson. Après avoir formaliser la syntaxe du langage à l'aide de la syntaxe BNF et de règles d'inférence, nous formaliserons la sémantique dénotationnelle ou opérationnelle sur ce langage. Ensuite, nous écrirons ces formalisations dans les 4 outils précédemment cités. Quand l'outil le permet, nous avons également réaliser des preuves de certaines propriétés sur ces sémantiques, comme le déterminisme ou l'équivalence entre sémantiques.
	
	Dans la suite de ce document, nous considérons que $n \in \mathbb{Z}$ et $x \in \mathbb{V}$ où $\mathbb{V}$ est un ensemble de variables.